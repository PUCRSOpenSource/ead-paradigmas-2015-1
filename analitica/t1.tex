
\documentclass[
	article,			% indica que é um artigo acadêmico
	11pt,				% tamanho da fonte
	oneside,			% para impressão apenas no verso. Oposto a twoside
	a4paper,			% tamanho do papel. 
	english,			% idioma adicional para hifenização
	brazil,				% o último idioma é o principal do documento
	]{abntex2}


\usepackage{cmap}				% Mapear caracteres especiais no PDF
\usepackage{lmodern}			% Usa a fonte Latin Modern
\usepackage[T1]{fontenc}		% Selecao de codigos de fonte.
\usepackage[utf8]{inputenc}		% Codificacao do documento (conversão automática dos acentos)
\usepackage{indentfirst}		% Indenta o primeiro parágrafo de cada seção.
\usepackage{nomencl} 			% Lista de simbolos
\usepackage{color}				% Controle das cores
\usepackage{graphicx}			% Inclusão de gráficos

\usepackage[brazilian,hyperpageref]{backref}	 % Paginas com as citações na bibl
\usepackage[alf]{abntex2cite}	% Citações padrão ABNT

\renewcommand{\backrefpagesname}{Citado na(s) página(s):~}

\renewcommand{\backref}{}

\renewcommand*{\backrefalt}[4]{
	\ifcase #1 %
		%Nenhuma citação no texto.%
	\or
		Citado na página #2.%
	\else
		Citado #1 vezes nas páginas #2.%
	\fi}%

\titulo{As superfícies quádricas}
\autor{Diego P. da Jornada\thanks{diego.jornada@acad.pucrs.br}}

\definecolor{blue}{RGB}{41,5,195}

\makeatletter
\hypersetup{
		pdftitle={\@title}, 
		pdfauthor={\@author},
    	pdfsubject={Modelo de artigo científico com abnTeX2},
	    pdfcreator={LaTeX with abnTeX2},
		pdfkeywords={abnt}{latex}{abntex}{abntex2}{atigo científico}, 
		colorlinks=true,       		% false: boxed links; true: colored links
    	linkcolor=blue,          	% color of internal links
    	citecolor=blue,        		% color of links to bibliography
    	filecolor=magenta,      		% color of file links
		urlcolor=blue,
		bookmarksdepth=4
}
\makeatother

\makeindex

\setlrmarginsandblock{4cm}{4cm}{*}
\setulmarginsandblock{4cm}{4cm}{*}
\checkandfixthelayout

\setlength{\parindent}{1.3cm}

\setlength{\parskip}{0.2cm}  % tente também \onelineskip

\SingleSpacing

\begin{document}

\frenchspacing 

\maketitle

\begin{resumoumacoluna}
    
		Este trabalho apresenta algumas superfícies quádricas bem como suas
		definições e representações no $R^3$.

 \vspace{\onelineskip}
 
 \noindent
\end{resumoumacoluna}

\textual

    \section*{Introdução}

		Dentro do escopo da disciplina de \emph{Geometria Analítica} o terceiro
		trabalho pode ser resumido da seguinte maneira: deve-se descrever as
		seguintes superfícies:  Elipsóides, Hiperbolóides e Parabolóides. Além
		das descrições também serão apresentados exemplos gráficos destas
		superfícies.

		O gráfico de uma equação no $R^3$ é chamado de superfície. Uma
		superfície quádrica é representada por uma equação do segundo grau, cuja
		forma geral é $ ax^2 + by^2 + cz^2 + dxy + exz + fyz + gx + hy + iz + j
		= 0$ onde os coeficientes \emph{a}, \emph{b}, \emph{...} e \emph{j} são
		números reais de forma que pelo menos um dos coeficientes \emph{a},
		\emph{b}, \emph{c}, \emph{d}, \emph{e} e \emph{f} é diferente de zero.

    \section{Elipsóides}

		Um elipsóide é uma superfície descritpa pelo movimento de uma elipse em
		torno de um eixo. Ao girarmos essa elipse em tonor do eixo $Oy$ obtemos
		o \emph{elipsóide de revolução}, caso o centro seja a origem então o
		elipsóide é definido pela seguinte equação:

			$$\frac{x^2}{a^2}+\frac{y^2}{b^2}+\frac{z^2}{c^2}=1$$

		Se o centro de um elipsóide é o ponto \emph{(h, k, l)} e seus eixos
		forem paralelos aos eixos coordenados a equação é do seguinte formato:

			$$\frac{(x^2-h)}{a^2}+\frac{(y^2-h)}{b^2}+\frac{(z^2-1)}{c^2}=1$$

	\section{Hiperbolóides} 

		Um hiperbolóide é uma superfície descrita pelo movimento de um segmento de
		reta em torno de um eixo. Cada ponto deste segmento de reta, quando em
		movimento gera um curva perpendicular ao eixo de rotação.

	\section{Parabolóides}

		Um parabolóide é uma superfície descrita pelo movimento de uma parábola ao
		longo de uma curva planar em torno de um eixo de rotação . Uma parábola é
		uma curva em duas dimensões em que todos os pontos satisfazem a equação $z =
		ax^2$. Esta parábola é chamada de geratriz da superfície.

	\section{Conclusão}
    
	\bibliography{bib}

	\nocite{boulos1997}
	\nocite{winterle2000vetores}
	\end{document}
